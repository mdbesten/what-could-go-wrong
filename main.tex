% adopted from "Riddles in the Dark" template on Overleaf
% incorporated ideas from "Mailmerged Conference Name Cards" template 
\documentclass[grid,avery5371]{flashcards}

% include font icons
\usepackage{fontawesome}

% format url
\usepackage{hyperref}

% specify the variant (AV, AI, LLM, etc) among available card decks.
\newcommand{\deck}{flow}

% read card info from external file
\usepackage{datatool}
%% The "database" is a comma-separated values (CSV) file.
%% The first line should contain the column headers, without space characters, e.g.
%% Name,JobTitle,Department
%%
%% If a field value contains a comma, then the field value needs to be surrounded with double quotes, e.g. 
%% John Smith,Lecturer,"School of Science, Mathematics and Engineering"
%%
%% Spreadsheet applications can usually export such a .csv file.
%%
%% If field values are expected to contain LaTeX special characters like $, &, then use \DTLloadrawdb{data}.csv instead.
\DTLloaddb{prompt}{prompts-\deck.csv}

\DTLloaddb{response}{responses-\deck.csv}

\usepackage[utf8]{inputenc}
\usepackage[T1]{fontenc}
\usepackage{ebgaramond}

\usepackage{multicol}

\geometry{headheight=12pt,footskip=4pt}
\usepackage{fancyhdr}
\pagestyle{fancy}
\fancyhf{}
\renewcommand{\headrulewidth}{0pt}
\chead{\small What could go wrong? ({\sc \deck} deck) }

\title{What could go wrong?}
% compiled by:
\author{Matthijs den Besten}
% based on work by Nik Martelaro
% see resitory at https://github.com/mdbesten/what-could-go-wrong-llm
\cardbackstyle[\Huge]{plain}
\cardfrontstyle[\large]{headings}



\begin{document}


\cardfrontfoot{What could go wrong?\qquad\faicon{creative-commons}}

\begin{flashcard}[\faicon{gamepad}\quad Gameplay Instructions]{
%\begin{turn}{180}
%\Rotatebox[origin=c]{90}{%
{\bf\small During the game}
\begin{multicols}{2}
\begin{enumerate}\tiny \setlength{\itemsep}{.5ex}
    \item Count the points 
    \item Take notes on ideas that you have not thought about before
    \item Some of the cards are causes, others are effects. Don’t worry about what the game designers intended with each card, go where the discussion is best.
    \item Some of the cards may be upsetting. (Such as, a person is abused.)
    \item It’s fine to take time to have discussion.
    \item Try not to get side tracked, though!
\end{enumerate}
\end{multicols}
%}
%\end{turn}{180}
}
{\bf\small What could go wrong?---Gameplay Instructions}
\begin{multicols}{2}
    \begin{enumerate}\tiny \setlength\itemsep{.2ex}
        \item All players draw 5 event cards (\faicon{reply}) from their stack
        \item Throw the dice to choose two players who will be the Card Czar, one will be the White Card Czar (he/she will act in the most ethical way possible) the other will be the Black Czar (he/she will act the opposite way). All other players don't know who is the Black or White Czar. 
        \item The two Card Czars then pull a black prompt card and read each card to the group. 
        \item All other players then put 1 white response card face down in their slot for each question. Each players is trying to answer "ethically" for the White Czar and "unethically" for the Black Czar.  
        \item The Card Czars then flip and read each white card out loud.
        \item The Card Czars then picks one of the white cards to further discuss. +1 point goes to the player whose card was chosen + 1 point goes to each player that guessed who was the White Czar and who was the Black Czar
        \item The group then discusses further what else could go wrong based on the chosen card. People can award +1 point anyone who makes a good point in discussion.
        \item After the discussion dissipates after a few minutes, the next player becomes the White Card Czar and clicks the “Deal” button. Each player then draws a new white card, so that they again have 5 cards in their hand..
    \end{enumerate}
    \end{multicols}
\end{flashcard}

% Adopted from http://www.unm.edu/~unmvclib/gamification/assessment/gameevaluationcriteria.pdf
\begin{flashcard}[\faicon{meh-o}\quad Game Evaluation Criteria]{
    \begin{itemize}\tiny \setlength{\itemsep}{.1ex}
        \item Category Biases
        \item Scale 1 to 7
        \begin{multicols}{3}
        \begin{itemize}
            \item Clarity
            \item Flow
            \item Balance
            \item Length
            \item Integrity
            \item Fun
        \end{itemize}
        \end{multicols}
        \begin{multicols}{2}
        \item Strongest Point
        \item Weakest Point
        \item One Change
        \item Comparable Games
        \end{multicols}
    \end{itemize}
    }
    {\bf\small What could go wrong?---Game Evaluation}
    \small
    \begin{tabular}{l|c|c|c|c|}
        & 1 & 3 & 5 & 7\\
    \hline
    {\em Clarity} & Opaque & Muddy & Transparent & Water-clear \\
    {\em Flow} & Cumbrous & Fraught & Smooth & Natural \\
    {\em Balance} & Broken & Fluky & Sensible & Fair\\
    {\em Length} & Wrong & Unfit & Apt & Perfect \\
    {\em Integrity} & Eristic & Erratic & Coherent & Sound \\
    {\em Fun} & Offputting & Boring & Engaging & Exciting \\
    \hline
    \end{tabular}
\end{flashcard}


\DTLforeach{prompt}{%
    %% Map each column header in your .csv file to a command
	\Description=PROMPT,%
    \Category=LABEL,%
    \Id=NUMBERS%
}{%%%  Start designing your output text!
\begin{flashcard}[\deck:\quad\Id\quad\DTLifnull{\Category}{}{\Category}]{%
%\begin{verse}[\versewidth]
\Description
%\end{verse}
}
\faicon{tasks}\quad{\sc \deck}\\
\end{flashcard}
}

\DTLforeach{response}{%
    %% Map each column header in your .csv file to a command
	\Description=RESPONSE,%
    \Category=LABEL,%
    \Id=NUMBERS%
}{%%%  Start designing your output text!
\begin{flashcard}[\deck:\quad\Id\quad\DTLifnull{\Category}{}{\Category}]{%
\Description
}
\faicon{reply}\quad{\sc \deck}\\
\end{flashcard}
}


\end{document}