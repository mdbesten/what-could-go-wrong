% adopted from "Riddles in the Dark" template on Overleaf
% incorporated ideas from "Mailmerged Conference Name Cards" template 
\documentclass[grid,avery5371]{flashcards}

% include font icons
\usepackage{fontawesome}

% format url
\usepackage{hyperref}

% specify the variant (AV, AI, LLM, etc) among available card decks.
\newcommand{\deck}{flow}

% read card info from external file
\usepackage{datatool}
%% The "database" is a comma-separated values (CSV) file.
%% The first line should contain the column headers, without space characters, e.g.
%% Name,JobTitle,Department
%%
%% If a field value contains a comma, then the field value needs to be surrounded with double quotes, e.g. 
%% John Smith,Lecturer,"School of Science, Mathematics and Engineering"
%%
%% Spreadsheet applications can usually export such a .csv file.
%%
%% If field values are expected to contain LaTeX special characters like $, &, then use \DTLloadrawdb{data}.csv instead.
\DTLloaddb{prompt}{prompts-\deck.csv}

\DTLloaddb{response}{responses-\deck.csv}

\usepackage[utf8]{inputenc}
\usepackage[T1]{fontenc}
\usepackage{ebgaramond}

\usepackage{multicol}

\geometry{headheight=12pt,footskip=4pt}
\usepackage{fancyhdr}
\pagestyle{fancy}
\fancyhf{}
\renewcommand{\headrulewidth}{0pt}
\chead{\small What could go wrong? ({\sc \deck} deck) }

\title{What could go wrong?}
% compiled by:
\author{Matthijs den Besten}
% based on work by Nik Martelaro
% see resitory at https://github.com/mdbesten/what-could-go-wrong-llm
\cardbackstyle[\Huge]{plain}
\cardfrontstyle[\large]{headings}

% Define commands for player scores
\newcommand{\playerscoreA}{0}
\newcommand{\playerscoreB}{0}
\newcommand{\playerscoreC}{0}

% Define commands to add points to player scores
\newcommand{\addpointsA}[1]{\addtocounter{playerscoreA}{#1}}
\newcommand{\addpointsB}[1]{\addtocounter{playerscoreB}{#1}}
\newcommand{\addpointsC}[1]{\addtocounter{playerscoreC}{#1}}

% Define commands to display player scores
\newcommand{\displayscores}{%
    \par
    Player A Score: \arabic{playerscoreA} points\par
    Player B Score: \arabic{playerscoreB} points\par
    Player C Score: \arabic{playerscoreC} points\par
}

\begin{document}

\cardfrontfoot{What could go wrong?\qquad\faicon{creative-commons}}

\begin{flashcard}[\faicon{info-circle}\quad Gameplay Instructions]{
  Please refer to the separate instructions file (instructions.md) for details on how to play the game.
}

% Adopted from http://www.unm.edu/~unmvclib/gamification/assessment/gameevaluationcriteria.pdf
\begin{flashcard}[\faicon{meh-o}\quad Game Evaluation Criteria]{
    \begin{itemize}\tiny \setlength{\itemsep}{.1ex}
        \item Category Biases
        \item Scale 1 to 7
        \begin{multicols}{3}
        \begin{itemize}
            \item Clarity
            \item Flow
            \item Balance
            \item Length
            \item Integrity
            \item Fun
        \end{itemize}
        \end{multicols}
        \begin{multicols}{2}
        \item Strongest Point
        \item Weakest Point
        \item One Change
        \item Comparable Games
        \end{multicols}
    \end{itemize}
    }
    {\bf\small What could go wrong?---Game Evaluation}
    \small
    \begin{tabular}{l|c|c|c|c|}
        & 1 & 3 & 5 & 7\\
    \hline
    {\em Clarity} & Opaque & Muddy & Transparent & Water-clear \\
    {\em Flow} & Cumbrous & Fraught & Smooth & Natural \\
    {\em Balance} & Broken & Fluky & Sensible & Fair\\
    {\em Length} & Wrong & Unfit & Apt & Perfect \\
    {\em Integrity} & Eristic & Erratic & Coherent & Sound \\
    {\em Fun} & Offputting & Boring & Engaging & Exciting \\
    \hline
    \end{tabular}
\end{flashcard}

\DTLforeach{prompt}{%
  %% Map each column header in your .csv file to a command
  \Description=PROMPT,%
  \Category=LABEL,%
  \Id=NUMBERS,%
  \promptdifficulty=\DTLfetch{prompt}{ID}{\deck:\quad\Id}{DIFFICULTY},%   <--- Added line to fetch difficulty
  \promptlanguage=\DTLfetch{prompt}{ID}{\deck:\quad\Id}{LANGUAGE}%    <--- Added line to fetch language
}{%%% Start designing your output text!
\begin{flashcard}[\deck:\quad\Id\quad\DTLifnull{\Category}{}{\Category} \quad Difficulty: \promptdifficulty \quad Language: \promptlanguage]{%  <--- Added difficulty and language info to card
%\begin{verse}[\versewidth]
\Description
%\end{verse}
}
\faicon{tasks}\quad{\sc \deck}\\
\end{flashcard}
}

\DTLforeach{response}{%
    \Description=RESPONSE,%
    \Category=LABEL,%
    \Id=NUMBERS,%
    \responsedifficulty=DIFFICULTY,%
    \responselanguage=LANGUAGE%
}{%
    \begin{flashcard}[\deck:\quad\Id\quad\DTLifnull{\Category}{}{\Category} \quad Difficulty: \responsedifficulty \quad Language: \responselanguage]{%
        \Description
    }
    \faicon{tasks}\quad{\sc \deck}\\
    \end{flashcard}
}

% Example usage of adding points to players' scores
\addpointsA{1} % Add 1 point to Player A's score
\addpointsB{2} % Add 2 points to Player B's score

% Display players' scores
\displayscores

\end{document}
